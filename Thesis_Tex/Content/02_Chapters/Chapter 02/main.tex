%%%%%%%%%%%%%%%%%%%%%%%%
%
%   Thesis template by Youssif Al-Nashif
%
%   May 2020
%
%%%%%%%%%%%%%%%%%%%%%%%%

\chapter{Literature Review}
\label{literature-review}


%---- Introduction


The introduction of graph kernels first arose in 2002, from the paper ``\textit{Diffusion kernels on graphs and other discrete structures}" ~\cite{kondor2002diffusion}. Since then, graph kernels have been used in a variety of fields, and modified to suit different problems. These developments have resulted in applications in biology, chemistry, social media analytics, and machine learning. Graph kernels have since formed into 3 large sub-groups: random walk kernels, sub-graph (also called graphlet) kernels, and tree-based methods. Within these three categories there are a multitude of modifications which include things such as edge or vertex attributes into the calculations ~\cite{vishwanathan2010graph}. The graph kernels of particular interest to this study are those which incorporate vertex attributes into the calculation of the kernel.  This inclusion of vertex attributes will allow the assignment of a word from a text document to a vertex. 

The use of graph kernels in text processing has been largely focused on applying them to a neural networks for NLP. However, a 2017 paper by Nikolentzos et.al, covered how their use first arrived at the idea to apply graph kernels to what they called ``graph-of-words" ~\cite{nikolentzos2017shortest}. They utilized a window width to construct a connections between words, effectively ``skipping" some words in between; this concept has also been called a ``skip-gram" network in other contexts ~\cite{cheng2006n}. Members of this research group, and others close to them, have produced software packages to perform some basic graph kernel methods. In addition, there have been a small number of papers from the group about the application of these methods to text mining ~\cite{sugiyama2018graphkernels}. Their work centered around use of neural networks for classification tasks. 



%%% TABLE %%%
\begin{table}[H]
\caption{Summary of Literature Review}
\centering
\begin{tabular}{ c c c c}
\hline
\hline
Topic & Author & Title & Year \\ [0.5ex]
\hline
Graph Kernels & Vishwanathan & Graph Kernels & 2010\\
Graph Kernels & Kriege et al. & A survey on graph kernels & 2020\\
Graph Kernels & Nikolentzos et al. & [graph kernels for doc. similarity] & 2017\\
Graph Kernels & Kondor and Lafferty &  Diffusion Kernels on Graphs [...] & 2002\\
Skip-Grams & Cheng et al. & From n-gram to skipgram [...] & 2006\\
Text Mining & Vazirgiannis et al. & GraphRep: [...] & 2018 \\
Application & Rosenfeld et al. & Kernel of Truth: [...] & 2020\\
Software & Silge and Robinson& tidytext &  2016\\
Software & Sugiyama et al. & graphkernels & 2018 \\
Software & Casardi et al. & igraph & 2013\\


\hline
\end{tabular}

\end{table}



Using a similar framework, the performance of the graph kernels as a pre-processing step for unsupervised learning will be assessed with data sources other than what the originators of the idea used. This assessment will result in validation, or disagreement, with their methods. The data sources will come from social media websites, which will serve as small document size examples, and technical reports from the National Highway and Transportation Safety Administration (NHTSA), which will serve as a larger document size example. These datasets, while very different in both writing style and length, provide an excellent opportunity to compare how performance, computation, and accuracy vary across differing types of text data. The NHTSA data was curated within a research group at Florida Polytechnic University this summer, and the process of data cleaning and transformation is in its final stages for initial exploratory data analysis and tests. This application would appeal to those within the Advanced Mobility Institute (AMI) as a way to text mine these technical documents and identify ``edge cases" within the reports, further supporting the efforts to provide closer to real life scenarios for testing and validation. The social media data will be collected off of reddit, and will remain focused on social health forums on the platform. This will be a continuation of work done at Florida Polytechnic University as well. By text mining these forums, there is an opportunity to better compare comments, threads, or communities at large, thus allowing for additional machine learning processes to take place. 


The work proposed here will center not only on unsupervised methods but also empirical studies on the effectiveness of graph kernels on these types of text data. The effects of varying window width, edge weighting, other graph structures, and document types will be considered. These contributions to the topic will aid in demonstrating the effectiveness on differing document types, and more importantly an exploration into using graph kernels as a preprocessing method for unsupervised learning methods. Using data sources I am familiar with, I will be building upon previous work in unsupervised text mining \cite{akioyamen2020framework}. Data and text mining have been a focus of my education at Florida Polytechnic and this project will be utilizing various skills I have learned through the program. In addition, this work will be relevant to my professional career as well.

%%%%%%%%%%






