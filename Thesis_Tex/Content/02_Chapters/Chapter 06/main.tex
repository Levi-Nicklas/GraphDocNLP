%%%%%%%%%%%%%%%%%%%%%%%%
%
%   Thesis template by Youssif Al-Nashif
%
%   May 2020
%
%%%%%%%%%%%%%%%%%%%%%%%%

\chapter{Future Work}

Future developments on this project would be focusing on adapting the methods to scale on high performance computing (HPC) environments. The methods currently do no do well as the number of graphs for comparison grows. While the methods do well as the size of the graph scales up, the issue of many graphs can be addressed through parallel processing for applications at scale. The code has already been structured to handle parallel processing, and could be further prepared for a HPC through containerization.\\

Additionally, supervised and semi-supervised learning methods could be explored on these datasets. Using the clustering results as labels would lead to a semi-supervised model. For classification applications, use of support vector machines in conjunction with these graph kernel methods is well documented, and would be a natural next step following this work. Once the datasets are labeled, it would be an easy study to complete. The NHTSA data set, with 48 observations, could be labeled with a number of different dependent variables that could be gleaned from the crash reports.\\

The reddit thread dataset, as large as it is, may be a good candidate for a semi-supervised application. The clustering results from this study could be applied in a supervised application on reddit threads from the same subreddit, and efficacy could be assessed.\\

Lastly, these methods used here consider edge histogram kernels exclusively, because they are computational cheaper than other methods, and exploration of the other kernel methods which utilize edge/vertex labels would be an important study to conduct. 
