%%%%%%%%%%%%%%%%%%%%%%%%
%
%   Thesis template by Youssif Al-Nashif
%
%   May 2020
%
%%%%%%%%%%%%%%%%%%%%%%%%

\section{Introduction}



\subsection{ Skip-grams}
\hspace*{0.5cm} As an alternative to natural language processing (NLP) methods, which are reliant on ``bag-of-words" methods, the methods used here utilize a graph representation of the text. Consider a bigram, a pair of two words\textemdash like ``hot dog" or ``peanut butter", these bigrams can be constructed for a text document where every pair or adjacent words is a bigram. The bigrams can then be used to make a graph, where each word is a vertex, and each bigram is an edge. This graph representation holds more context than the bag-of-words methods; for example seeing the words ``cake" and ``carrot" in a bag of words may not show that ``carrot cake" was the real intent of the text. This is an important concept for modeling text, as we should strive to achieve a representation of the text that makes for effective modeling that will capture the true meaning of the text in question. Keeping this in mind, with the example of ``carrot cake", what about the idiom ``beating a dead horse"? Each word individually may mean something other than the idiom. Even the bigrams ``beating dead" and ``dead horse" do not capture what the idiom means. We can expand the number of words in the n-gram to be 3 or 4 words, or alternatively, we can make more ``edges" or connect more words. We can connect words that are not immediately adjacent but perhaps within $k$ words away. These bigrams that appear within $k$ words of each other are called ``skip-grams". The skip-gram allows to capture context of larger sequences of words since the graph representation will show how the $k$ wide neighborhood of words was connected. In the idiom example, using skip-grams with window width $k = 2$, and removing common words (e.g. ``a", ``at", ``the"), will produce a graph like: 

$$
E(G) = \{
\text{beat}  \longleftrightarrow \text{dead}, 
\text{dead}  \longleftrightarrow \text{horse}, 
\text{horse}  \longleftrightarrow \text{beat} \}
$$

This graph representation contains a cycle, of length 3, where most native english speakers will identify the meaning behind the graph representation. As ideas, idioms, figures of speech, and other concepts (that may be explained in a non-literal fashion) grow in size as they include more words, it becomes more difficult to capture the meaning behind the text. However, leveraging the concept of a skip gram can produce such a rich graph representation of the text that the original meaning is more likely to be preserved. \\

\subsection{ Graph Kernels}
The next natural question is, ``how can we compare these graph representations?", and we address this with graph kernel methods. These methods are generally used to compare the similarity of graphs. These use of a graph kernel to compare graphs was first published in 2003, and since then various applications and adaptations have been made to the methods. In the case of text mining, the graph kernel must assess vertex labels \textemdash if one intends to map words to vertices, otherwise they will be assessing the topology alone. In this study, the Edge-Histogram kernel is the kernel used to compute similarity. This kernel was chosen as it uses labels on the graph structure, and is not as computationally intensive as other methods \cite{sugiyama2015halting}. In the specific implementation used for these studies, the computation time was the shortest when compared with other kernel methods like: graphlet, random walk, and Weisfeiler-Lehman kernel \cite{sugiyama2015halting}. Since the data sets of concern in the studies feature either large graphs or a large number of graphs, the kernel had to be cheap computationally.\\

To compute and edge histogram kernel on two graphs, $G_1$ and $G_2$, first define the set of edges $E_i =\{ (u_1,v_1), (u_2,v_2), ... , (u_n,v_n) \}$ where $(u_n,v_n)$ is the $n$-th edge connecting $u_n$ to $v_n$. Then the edge label histogram is defined to be $\vec{g} = \{ g_1, g_2, ... , g_s\}$, so that $g$ contains the number of times each edge label appeared. In the case of graphical representations of text, the number of times a skip-gram appears is not considered; it either appeared or did not. For this reason, a manhattan distance is chosen, as opposed to a euclidean or similar distance metric. The kernel is then the sum or the product of each element in the $g$ vectors for each $G_1$ and $G_2$ in the case of a linear kernel \cite{sugiyama2015halting}.



%https://papers.nips.cc/paper/2015/file/31b3b31a1c2f8a370206f111127c0dbd-Paper.pdf

\subsection{Using Kernel for Clustering}

The output of the kernel is useful for a variety of tasks. Some other popular applications have included classification with support vector machines, which are popular with other kernel methods. In this case, the kernel is used for unsupervised clustering. Within the kernel matrix, $K$, the entry $k_{i,j}$ represents the similarity between graphs $i$ and $j$. This matrix which contains measures of similarity between points can be used as a distance matrix for hierarchical clustering. Before using the graph kernel as a distance matrix, normalization or standardization takes place, and principal component analysis may be used. The end result is each row is a single graph-document being described by its similarity to all the other graphs, which are the column values. Once the values are transformed or rotated by preprocessing methods, the points are just represented by their similarity to one another, but in a transformed space. Various hyper parameters can be tuned for successful clustering; the graph kernel has a parameter that can be tuned, and the hierarchical clustering can be tried with differing types of linkage.


%\subsection{Kernel Density Estimation Clustering for Linear Kernel}

%In addition to hierarchical clustering, a method was developed to find potential clusters based on the kernel similarity measure, but while measuring similarity to a single graph. For example, we can compare how similar graphs $B$ and $C$ without computing their similarity, by comparing how similar they are to graph $A$; this extension of transitive property logic allows for focusing on the similarity of the graph list as it relates to just one graph. 










