%%%%%%%%%%%%%%%%%%%%%%%%
%
%   Thesis template by Youssif Al-Nashif
%
%   May 2020
%
%%%%%%%%%%%%%%%%%%%%%%%%

\section{Introduction}

As an alternative to natural language processing (NLP) methods, which are reliant on "bag-of-words" methods, the methods used here utilize a graph representation of the text. Consider a bigram, a pair of two words\textemdash like "hot dog" or "peanut butter", these bigrams can be constructed for a text document where every pair or adjacent words is a bigram. The bigrams can then be used to make a graph, where each word is a vertex, and each bigram is an edge. This graph representation holds more context than the bag-of-words methods; for example seeing the words "cake" and "carrot" in a bag of words may not show that "carrot cake" was the real intent of the text. This is an important concept for modeling text, as we should strive to achieve a representation of the text that makes for effective modeling that will capture the true meaning of the text in question. Keeping this in mind, with my example of "carrot cake", what about the idiom "beating a dead horse"? Each word individually may mean something other than the idiom. Even the bigrams "beating dead" and "dead horse" do not capture what the idiom means. We can expand the number of words in the n-gram to be 3 or 4 words, or alternatively, we can make more "edges" or connect more words. We can connect words that are not immediately adjacent but perhaps within $k$ words away. These bigrams that appear within $k$ words of each other are called "skip-grams". The skip-gram allows to capture context of larger sequences of words since the graph representation will show how the $k$ wide neighborhood of words was connected. In the idiom example, using skip-grams with window width $k = 2$, and removing common words (e.g. "a", "at", "the"), will produce a graph like: 

$$
E(G) = \{
\text{beat}  \longleftrightarrow \text{dead}, 
\text{dead}  \longleftrightarrow \text{horse}, 
\text{horse}  \longleftrightarrow \text{beat} \}
$$

%--- Any subsection will go here.
